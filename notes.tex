\documentclass[fontset=none,oneside]{book}
\usepackage[UTF8,heading=true,scheme=chinese]{ctex}
\usepackage{fontspec}
\usepackage{amsmath}
\usepackage{amssymb}
\usepackage{mathrsfs}
\usepackage{cite}
\usepackage{array}
\usepackage{graphicx}
\usepackage{bm}
\usepackage{algorithm}
\usepackage{algpseudocode}
\usepackage[tight,footnotesize]{subfigure}
\usepackage{cite}
\usepackage{url}
\usepackage{balance}
\usepackage{color}
\usepackage{framed}
\usepackage{enumitem}
\usepackage{subeqnarray}
\usepackage{makeidx}
\usepackage[top=2.54cm,bottom=2.54cm,left=2.54cm,right=2.54cm]{geometry}

\definecolor{shadecolor}{gray}{0.85}
\pagestyle{headings}
\ctexset{fontset=mac}
\ctexset{space=true}
\ctexset{section={
	name={第,节},
	number=\arabic{chapter}.\arabic{section},
}
}

%%
%% This is file `xeCJK-fonts.def',
%% generated with the docstrip utility.
%%
%% The original source files were:
%%
%% install/buptgraduatethesis.dtx  (with options: `def')
%% 
%% Copyright (C) 2013-2015 by Xianling Wang <rioxwang@foxmail.com>
%% 
%% This file is part of the BUPTGraduateThesis project.
%% 

\setCJKmainfont[BoldFont={SimHei}, ItalicFont={KaiTi}]{SimSun}
\setCJKsansfont{SimHei}
\setCJKmonofont{KaiTi}
\setCJKfamilyfont{song}{SimSun}
\setCJKfamilyfont{hei}{SimHei}
\setCJKfamilyfont{fs}{FangSong}
\setCJKfamilyfont{kai}{KaiTi}
\setCJKfamilyfont{li}{LiSu}
\setCJKfamilyfont{you}{YouYuan}
\setCJKfamilyfont{asong}{Adobe Song Std}
\setCJKfamilyfont{ahei}{Adobe Heiti Std}
\setmainfont{Times New Roman}
\setsansfont{Arial}
\setmonofont{Courier New}
\newcommand{\song}{\CJKfamily{song}}    % 中易宋体
\def\songti{\song}
\newcommand{\hei}{\CJKfamily{hei}}      % 中易黑体
\def\heiti{\hei}
\newcommand{\fs}{\CJKfamily{fs}}        % 中易仿宋体
\def\fangsong{\fs}
\newcommand{\kai}{\CJKfamily{kai}}      % 中易楷体
\def\kaishu{\kai}
\newcommand{\asong}{\CJKfamily{asong}}  % Adobe宋体
\def\asongti{\asong}
\newcommand{\ahei}{\CJKfamily{ahei}}    % Adobe黑体
\def\aheiti{\ahei}

%\makeindex
\renewcommand{\contentsname}{目 \quad 录}
\begin{document}
%\title{Lecture Notes}
\title{Mathematic Notes}
\author{邵华}
\maketitle
\tableofcontents
%\printindex
\chapter{矩阵计算}
\section{文中符号}
\begin{itemize}
\item 用大写字母$\bf A$表示矩阵,小写字母$\bf a$表示列向量,a表示实数或者复数标量
\item $+,-,*$分别表示矩阵加,减乘
\item $\bf A^T$表示$\bf A$的转置
\item $\bf A^H$表示$\bf A$的共轭转置,如果$\bf A$是实矩阵,$\bf A^T=\bf A^H$
\item $\bf A^{-1},A^{\#},A^{+}$分别表示逆,一般逆和伪逆
\item $.*,./$表示逐元素的乘法和除法
\item $|\bf A|,||\bf A||_F$代表行列式和Frobenius范数
\item $||\bf a||$代表欧几里得范数
\item $|a|$代表$a$的绝对值
\end{itemize}
\section{矩阵推导结论}
\begin{equation}
\xi_i=\frac{||\rm{H}\rm{T}||_{i,i}^2}{[\rm{HTT^HH^H}]_{i,i}}
\end{equation}
将${\bf HT}$进行SVD分解
\begin{equation}
{\bf HT=U}\Lambda{\bf V^H }
\end{equation}
则有:
\begin{equation}
\xi_i=\frac{||{\bf HT}||_{i,i}^2}{[\bf{HTT^HH^H}]_{i,i}}=\frac{|u_i^H\Lambda v_i|^2}{|u_i^H\Lambda^2v_i|}
\end{equation}
其中,$u_i,v_i$分别是$\mathbf{U^H,V^H}$的第$i$列
\begin{shaded}
\noindent
\emph{
Cauchy-Schwarz不等式:
\begin{equation}
|\langle x,y \rangle|^2\leq \langle x,x \rangle\cdot\langle y,y \rangle
\end{equation}
}
\emph{Cauchy-Schwarz积分不等式:
\begin{equation}
(\int_{a}^{b}f(x)g(x)dx)^2\leq \int_{a}^{b}f^2(x)dx\int_{a}^{b}g^2(x)dx
\end{equation}
}
\emph{$H\ddot{o}lder$不等式,当$p,q\geq 1$,且$\frac{1}{p}+\frac{1}{q}=1$
\begin{equation}
|\sum\limits_{i=1}^{n}a_ib_i|\leq (\sum\limits_{i=1}^{n}|a_i|^p)^{1\over p}(\sum\limits_{i=1}^{n}|b_i|^q)^{1\over q}
\end{equation}
}
\emph{Minkowski积分不等式:
\begin{equation}
(\int_{a}^{b}|f(x)g(x)|^pdx)^{\frac{1}{p}} \leq (\int_{a}^{b}|f(x)|^pdx)^{\frac{1}{p}} (\int_{a}^{b}|g(x)|^pdx)^{\frac{1}{p}}
\end{equation}
}
\emph{当p=q=2时候,Minkowski积分不等式和$H\ddot{o}lder$不等式退化成Cauchy不等式。}
\end{shaded}

\begin{shaded}
Jesen不等式,给出了积分的凸函数值和凸函数积分值之间的关系:若$f(x)$是
区间$[a,b]$的凸函数,对任意的$x_{i},x_{2},x_{3},\cdots,x_{n}\in (a,b)$,
有
\begin{equation}
\label{eq:1}
f(\frac{x_{n}+x_{2}+\cdots+x_{n}}{n})\leq \frac{1}{n}[f(x_{1})+f(x_{2})+\cdots+f(x_{n})]
\end{equation}
\end{shaded}

\begin{shaded}
\begin{equation}
\label{eq:2}
{\bf Tr[AA^{H}]=||vec(A)||^{2}=||vec{(A^{H})||^{2}}}
\end{equation}
\begin{equation}
\label{eq:3}
\sum_{i}\lambda_{i}||\mathbf{A}e_{i}||^{2}= \mathbf{Tr} \{\mathbf{A} \Lambda \mathbf{A^{H}}
    \}
\end{equation}
\end{shaded}

\section{矩阵微积分}
\subsection{符号表示}
\begin{shaded}
\begin{itemize}
\item $d/dx(\bf y)$是一个向量,第$i$个元素为$dy(i)/dx$
\item $d/d{\bf x}(y)$是一个向量,第$i$个元素为$dy/dx(i)$
\item $d/d{\bf x}({\bf y^T})$是一个矩阵,第$(i,j)$个元素为$dy(j)/dx(i)$
\item $d/dx({\bf Y})$是一个矩阵,第$(i,j)$个元素为$dy(i,j)/dx$
\item $d/d{\bf X}(y)$是一个矩阵,第$(i,j)$个元素是$dy/dx(i,j)$
\end{itemize}
\end{shaded}
\subsection{线性乘积的导数}
\begin{itemize}
\item $d/dx({\bf AYB})={\bf A}*d/dx({\bf Y})*{\bf B}$
\item $d/dx({\bf Ay})={\bf A}*d/dx({\bf y})$
\item $d/d{\bf x}({\bf x^TA})={\bf A}$
\item $d/d{\bf x}({\bf x^T})={\bf I}$
\item $d/d{\bf x}({\bf x^Ta})=d/d{\bf x}({\bf a^Tx})={\bf a}$
\item $d/d{\bf X}({\bf a^TXb})={\bf ab^T}$
\item $d/d{\bf X}({\bf a^TXa})=d/d{\bf X}({\bf a^TX^Ta})={\bf aa^T}$
\item $d/d{\bf X}({\bf a^TX^Tb})={\bf ba^T}$
\item $d/dx({\bf YZ})={\bf Y}*d/dx({\bf Z})+d/dx({\bf Y})*{\bf Z}$
\end{itemize}
\subsection{二次乘积的导数}
\begin{itemize}
\item $d/d{\bf x}{\bf (Ax+b)^TC(Dx+e)}={\bf A^TC(Dx+e)+D^TC^T(Ax+b)}$
\item $d/d{\bf x}{\bf(x^TCx)}={\bf (C+C^T)x}$
\item $d/d{\bf x}{\bf(x^TCx)}=2Cx$
\item {\color{red} [C:symmetric]}$d/d{\bf x}{\bf x^Tx}=2x$
\item $d/d{\bf x}{\bf (Ax+b)^T(Dx+e)}={\bf A^T(Dx+e)+D^T(Ax+b)}$
\item $d/d{\mathbf{x}}\mathbf{(Ax+b)^T(Ax+b)=2A^T(Ax+b)}$
\item {\color{red}[C:symmetric]}$d/d{\bf x}\mathbf{(Ax+b)^TC(Ax+b)=2A^TC(Ax+b)}$
\item $d/d{\bf X}\mathbf{(a^TX^TXb)=X(ab^T+ba^T)}$
\item $d/d{\bf X}\mathbf{(a^TX^TXa)=2Xba^T}$
\item $d/d{\bf X}\mathbf{(a^TX^TCXb)=C^TXab^T+CXba^T}$
\item $d/d{\bf X}\mathbf{(a^TX^TCXa)=(C+C^T)Xaa^T}$
\item {\color{red} [C:Symmetric]} $d/d{\bf X}\mathbf{(a^TX^TCXa)=2CXaa^T}$
\item $d/d{\bf X}\mathbf{((Xa+b)^T)C(Xa+b)=(C+C^T)(Xa+b)a^T}$
\end{itemize}
\subsection{三次乘积的导数}
\begin{equation}
d/d{\bf x}\mathbf{(x^TAxx^T)=(A+A^T)xx^T+x^TAxI}
\end{equation}
\subsection{逆矩阵的导数}
\begin{equation}
d/d{\bf x}\mathbf{(Y^{-1})=-Y^{-1}}d/dx\mathbf{(Y)Y^{-1}}
\end{equation}
\subsection{迹的导数}
\begin{itemize}
\item $d/d{\bf X}\mathbf{(tr(X))=I}$
\item $d/d\bf X(tr\mathbf{(X^k)}=\mathbf{k(X^{k-1})^T}$
\item $d/d\bf
    X(tr\mathbf{(AX^{k})})=\mathbf{\sum\limits_{r=0:k-1}(X^{r}AX^{k-r-1})^{T}}$
\item $d/d\bf X(tr(AX^{-1}B))=-(X^{-1}BAX^{-1})^{T}$
\item $d/d{\bf X (tr(AX^{-1}))}=d/d{\bf X
    (tr(X^{-1}A))=-X^{-T}A^{T}x^{-T}}$
\item $d/d{\bf X(tr(A^{T}XB^{T}))}=d/d{\bf X(tr(BX^{T}A))=AB}$
\item $d/d{\bf X(tr(XA^{T}))}=d/d{\bf X(tr(X^{T}A))}=d/d{\bf
    X(tr(A^{T}X))}=d/d{\bf X(tr(AX^{T}))}={\bf A}$
\item $d/d{\bf X(tr(AXBX^{T}))=A^{T}XB^{T}+AXB}$
\item $d/d{\bf X(tr(XAX^{T}))=X(A+A^{T})}$
\item $d/d{\bf X(tr(X^{T}AX))=X^{T}(A+A^{T})}$
\item $d/d{\bf X(tr(AX^{T}X))=(A+A^{T})X}$
\item $d/d{\bf X(tr(AXBX))=A^{T}X^{T}B^{T}+B^{T}X^{T}A^{T}}$
\item {\color{red}[C:symmetric]}$d/d{\bf X(tr((X^{T}CX)^{-1}A)}=d/d{\bf
  X(tr(A(X^{T}CX)^{-1}))}=\bf
-(CX(X^{T}CX)^{-1})(A+A^{T})(X^{T}CX)^{-1}$
\item {\color{red}[B,C:Symmetric]}
\item $d/d{\bf
    X(tr((X^{T}CX)^{-1}(X^{T}BX))}=d/d{\bf
    X(tr((X^{T}BX)(X^{T}CX)^{-1}}={\bf -2(CX(X^{T}CX)^{-1}X^{T}BX(X^{T}CX)^{-1}+2BX(X^{T}CX)^{-1}}$
\end{itemize}
\subsection{行列式的导数}
\begin{itemize}
\item $d/d{\bf X(det(X))}=d/d{\bf X(det(X^{T}))}={\bf det(X)*X^{-T}}$
\item $d/d{\bf X(det(AXB))}={\bf det(AXB)*X^{-T}}$
\item $d/d{\bf X(ln(det(AXB)))}={\bf X^{-T}}$
\item $d/d{\bf X(det(X^{k}))=k*det(X^{k})*X^{-T}}$
\item $d/d{\bf X(ln(det(X^{k})))=kX^{-T}}$
\item {\color{red}[Real]} $d/d{\bf X(det(X^{T}CX))}={\bf
    det(X^{T}CX)*(C+C^{T})X(X^{T}CX)^{-1}}$
\item {\color{red} [C:Real,Symmetric]}$d/d{\bf
    X(det(X^{T}CX))=2det(X^{T}CX)*CX(X^{T}CX)^{-1}}$
\item {\color{red}[C:Real,Symmetric]}$d/d{\bf X(ln(det(X^{T}CX)))=2CX(X^{T}CX)^{-1}}$
\end{itemize}
\subsection{雅可比}
如果$\bf y$是$\bf x$的函数,则有$d{\bf y^T}/d{\bf x}$是$\bf y$相对于$\bf x$的雅可比矩阵。
$|d{\bf y^T}/d{\bf x}|$称为$\bf y$相对于$\bf x$的雅可比行列式,它主要用在积分中变换积分变量:
\begin{equation}
\int (f({\bf y})d{\bf y})=\int(f(\mathbf{y(x)})|d{\bf y^T}/d{\bf x}|dx)
\end{equation}
\subsection{Hessian矩阵}
如果$f$是$x$的函数,则对称矩阵$d^{2}f/dx^{2}=d/dx^{T}(df/dx)$被称为函
数f(x)的Hessian矩阵。对于$df/d{\bf x}=0$的点,若Hessian矩阵是正定,负
定和不定的情况,分别为最小值,最大值,和鞍点。
\begin{shaded}
\begin{itemize}
\item $d^{2}/d{\bf x^{2}(AX+B)^{T}C(Dx+e)=A^{T}CD+D^{T}C^{T}A}$
\item $d^{2}/d{\bf x^{2}(a^{T}x)=0}$
\item $d^{2}/d{\bf x^{2}(x^{T}x)=2I}$
\item $d^{2}/d{\bf x^{2}(x^{T}Cx)=C+C^{T}}$
\item $d^{2}/d{\bf x^{2}(Ax+b)^{T}(Dx+e)=A^{T}D+D^{T}A}$
\item $d^{2}/d{\bf x^{2}(Ax+b)^{T}(Ax+b)=2A^{T}A}$
\item {\color{red}[C:Symmetric]} $d^{2}/d{\bf x^{2}(Ax+b)^{T}C(Ax+b)=2A^{T}CA}$
\end{itemize}
\end{shaded}
\section{二阶锥规划}
\subsection{基本形式\cite{Lobo:1998aa}}
\begin{shaded}
\begin{equation}
\label{eq:4}
\begin{split}
\mathbf{minimize}~ &f^{T}(x) \\
\mathbf{subject~to}~& ||A_{i}x+b_{i}|| \leq c_{i}^{T}x+d_{i},~i=1,2,\cdots,N
\end{split}
\end{equation}\index{二阶锥规划基本形式}
\end{shaded}
其中,$x\in \mathbb{R}^{n}$是优化变量,优化问题的参数为$f\in
\mathbb{R}^{n},A_{i}\in \mathbb{R}^{(n_{i}-1)\times n},b_{i}\in
\mathbb{R}^{n_{i}-1},c_{i}\in \mathbb{R}^{n},d_{i}\in \mathbb{R}$。范
数表示欧几里得范数,即:$||u||=(u^{T}u)^{1/2}$。约束
\begin{equation}
\label{eq:5}
||A_{i}x+b_{i}||\leq c_{i}^{T}+d_{i}
\end{equation}
为维度$n_{i}$的二阶锥约束。因为标准锥,或者k维单位二阶凸锥被定义为:
\begin{equation}
\label{eq:6}
\mathscr{C}_{k}=\left\{\begin{bmatrix}u\\t\end{bmatrix}|u\in\mathbb{R}^{k-1},t\in\mathbb{R},||u||\leq t\right\}
\end{equation}
又被称做二次冰锥,活着洛伦兹锥。由于二阶锥约束的点集是单位二阶锥在仿射
映射下的原象:
\begin{equation}
\label{eq:7}
||A_{i}x+b_{i}||\leq c_{i}^{T}x+d_{i} \Longleftrightarrow \begin{bmatrix}A_{i}\\c_{i}^{T}\end{bmatrix}x+\begin{bmatrix}b_{i}\\d_{i}\end{bmatrix}\in\mathscr{C}_{n_{i}}
\end{equation}
因此是一个凸约束。故SOCP是一个凸优化问题。
\subsection{半正定问题}
二阶锥规划可以转化成如下形式的半正定问题:
\begin{equation}
\label{eq:8}
\begin{split}
\mathbf{minimize}~&f^{T}x \\
\mathbf{subject~to}~&\begin{bmatrix}(c_{i}^{T}x+d_{i})I &
  A_{i}x+b_{i}\\ (A_{i}x+b_{i})^{T} & c_{i}^{T}x+d_{i} \end{bmatrix}
\succeq 0,~i=1,\cdots,N
\end{split}
\end{equation}
然而,通过求解半正定规划来求解二次锥优化并不是一个好的办法,因为内点法
对于SOCP问题,比SDP问题有更好的收敛速度。
\subsection{能够变换为SOCP问题的形式}
\subsubsection{二次约束的二次规划}
\subsubsection{范数的和与最大值问题}
\subsubsection{双曲约束问题}
\subsubsection{矩阵泛函问题}
\subsubsection{可表示为二阶锥的函数和集合}

\chapter{最优化问题与KKT条件}
\section{优化问题与对偶函数、对偶问题}
考虑如下的最优化问题:
\begin{equation}
\label{eq:9}
\begin{split}
\mathbf{minimize ~} & f_{0}(x) \\
\mathbf{subject~to~}& f_{i}(x)\leq 0,~i=1,2,\cdots,m \\
                    & h_{i}(x)=0,~i=1,2,\cdots,p
\end{split}
\end{equation}
其自变量为$x\in \mathbf{R}^{n}$,设问题的定义域为
$\mathcal{D}=\bigcap\limits_{i=0}^{m}\mathbf{dom} f_{i} \cap
\bigcap\limits_{i=1}^{p}\mathbf{dom} h_{i}$,优化问题的最优值为$p^{*}$。
注意,这里并没有假设\eqref{eq:9}是一个凸优化问题。

问题\eqref{eq:9}的Lagrange函数
$L:\mathbf{R}^{n}\times\mathbf{R}^{m}\times\mathbf{R}\rightarrow\mathbf{R}$
为:
\begin{equation}
\label{eq:10}
L(x,\lambda,v)=f_{0}(x)+\sum\limits_{i=1}^{m}\lambda_{i}f_{i}(x)+\sum\limits_{i=1}^{p}v_{i}h_{i}(x)
\end{equation}

{\bf{Lagrange 对偶函数}}$g$为
$g:\mathbf{R}^{m}\times\mathbf{R}^{p}\rightarrow\mathbf{R}$,关于
{\bf{Lagrange函数}}关于$x$的最小值:
\begin{equation}
\label{eq:11}
g(\lambda,v)=\mathbf{inf}_{x\in\mathcal{D}}L(x,\lambda,v)
\end{equation}
\begin{shaded}
由于{\bf{Lagrange对偶函数}}是关于$(\lambda,v)$的仿射函数的逐点下确界,所
以,即使原问题不是凸的,对偶函数也是{\bf{凹函数}}。
\end{shaded}
于是,有$\lambda\geq 0, v$,下面式子成立:
\begin{equation}
\label{eq:12}
g(\lambda,v)\leq p^{*}
\end{equation}
即$g(\lambda,v)$是愿问题最优解的一个下界。

Lagrange对偶问题:
\begin{equation}
\label{eq:13}
\begin{split}
\mathbf{maximize ~ }& g(\lambda,v) \\
\mathbf{subject~to~}& \lambda\geq 0
\end{split}
\end{equation}
设Lagrange对偶问题具有可行解,即满足$\lambda\leq 0$和
$g(\lambda,v)>-\infty$的一组$(\lambda,v)$。称$(\lambda^{*},v^{*})$为
对偶最优解,或者最优Lagrange乘子,对应的最优值为$d^{*}$。根据定义,如
下额式子成立:
\begin{equation}
\label{eq:14}
d^{*}\leq p^{*}
\end{equation}
定义差值$p^{*}-d^{*}$为原问题的最有对偶间隙。
\section{弱对偶性与强对偶性}
\eqref{eq:14}称为弱对偶性。若$p^{*}=d^{*}$,则称强对偶性成立。

对于一般的情况,强对偶性不成立。但是如果原问题是凸问题,则强对偶性通常
(但不总是)成立。强对偶性成立的条件:1.原问题是凸问题,2.Slater条件成
立:存在一点$x\in \mathbf{relint}~\mathcal{D}$,使得下面式子成立了:
\begin{equation}
\label{eq:15}
f_{i}(x)<0,~i=1,\cdots,m, ~ Ax=b
\end{equation}
满足上面式子的点又被称为是严格可行点,这是因为不等式约束严格成立。

若化的Slater条件:若不等式约束函数$f_{i}$中有写函数是仿射函数,则仿射
不等式不需要严格成立:
\begin{equation}
\label{eq:16}
f_{i}(x)\leq 0,i=1,2\cdots,k,~f_{i}(x)<0,~i=k+1,\cdots,m,~Ax=b
\end{equation}
其中,$f_{1},f_{2},\cdots,f_{k}$是仿射的。
\begin{shaded}
注意到当所有的约束条件都是线性等式或者不等式,并且$\mathbf{dom} f_{0}$
是开集时,条件\eqref{eq:16}就是可行性条件。
\end{shaded}
\section{互补松性和KKT条件}
令$x^{*}$和$(\lambda^{*},v^{*})$分别是原问题和对偶问题的最优解,则有:
\begin{equation}
\label{eq:17}
\begin{split}
f_{0}(x^{*})&=g(\lambda^{*},v^{*}) \\
       & = \mathbf{inf}_{x}L(x^{*},\lambda^{*},v^{*})\\
       & \leq
       f_{0}(x^{*})+\sum\limits_{i=1}^{m}\lambda_{i}^{*}f_{i}(x^{*})+\sum\limits_{i=1}^{p}v_{i}^{*}h_{i}(x^{*})
       \\
       & \leq f_{0}(x^{*})
\end{split}
\end{equation}
第一个等式说明最优对偶间隙为零,第二个是对偶函数的定义,第三个不等式根
据Lagrange函数关于$x$求下确界小于等于其在$x=x^{*}$处的值求得来。由此可
得一个重要的结论:
\begin{equation}
\label{eq:18}
\sum\limits_{i=1}^{m}\lambda_{i}^{*}f_{i}(x^{*})=0
\end{equation}
\begin{shaded}
由于每一项都非正,因此有:
\begin{equation}
\label{eq:19}
\lambda_{i}^{*}f_{i}(x^{*})=0,~i=1,\cdots,m.
\end{equation}
上面的条件称为{\bf{互补松弛性.}}
\end{shaded}

Karush-Kuhn-Tucker(KKT)条件:
\begin{shaded}
\begin{equation}
\label{eq:20}
\begin{split}
f_{x}(x^{*}) & \leq 0,~ i=1,\cdots,m \\
h_i(x^{*}) &=0,~ i=1,\cdots,m \\
\lambda_{i}^{*}&\geq 0,~i=1,\cdots,m \\
\lambda_{i}^{*}f_{i}(x^{*})&=0,~i=1,\cdots,m \\
\Delta f_{0}(x^{*})+\sum\limits_{i=1}^{m}\lambda_{i}^{*}\Delta
f_{i}(x^{*})+\sum\limits_{i=1}^{*}v_{i}^{*}\Delta h_{i}(x^{*})&=0
\end{split}
\end{equation}
\end{shaded}
若某个凸优化问题具有可微的目标函数和约束函数,且其满足Slater条件,那么
KKT条件是最优性的充要条件:Slater条件意味着最优对偶间隙为零,且对偶最
优解可以达到,因此x是原问题最优解,当且仅当存在 $(\lambda,v)$,二者满
足KKT条件。
\chapter{概率和分布相关}
\section{正态分布、Gamma分布、卡方分布、Nakagami-m分布}
\subsection{Basic: Exponential Distributon}
\noindent PDF:
\begin{equation}
\label{eq:35}
f(x;\lambda)=\lambda e^{-\lambda x} \quad x\geq 0
\end{equation}
$\lambda$ is the \textbf{Rate} parameter, while $\beta=1/\lambda$ is the
\textbf{Scale} parameter. \\
CDF:
\begin{equation}
\label{eq:36}
F(x;\lambda)=1-e^{-\lambda x} \quad x\geq 0
\end{equation}
$\mathbb{E}[X]=\frac{1}{\lambda}=\beta$,
$\textbf{Var}[X]=\frac{1}{\lambda^{2}}$ \\
Properties:\\
\begin{itemize}
\item $X_{i}\sim \rm{Exp}(\lambda)$, then
  $\min(X_{1},X_{2},\cdots,X_{n})\sim\rm{Exp}(n\lambda)$
\item $X_{i}\sim \rm{Exp}(\lambda)$, then
  $X_{1}+X_{2}+\cdots+X_{k}\sim\rm{Erlang}(k,\lambda)=Gamma(k,\lambda^{-1})\quad
  (k,\theta)-form$
\end{itemize}
\subsection{Normal Distribution}
\noindent Notation:
\begin{equation}
\label{eq:23}
\mathcal{N}(\mu,\sigma^{2}) 
\end{equation}
PDF:
\begin{equation}
\label{eq:21}
f(x;\mu,\sigma)=\frac{1}{\sigma\sqrt{2\pi}}e^{-\frac{(x-\mu)^{2}}{2\sigma^{2}}}
\end{equation}
CDF:
\begin{equation}
\label{eq:22}
F(x;\mu,\sigma)=\frac{1}{2}[1+\mathrm{erf}(\frac{(x-\mu)^{2}}{\sigma\sqrt{2}})]
\end{equation}
运算:
\begin{itemize}
\item $X_{1} \sim
  \mathcal{N}(\mu_{1},\sigma_{1})$,$X_{2}\sim\mathcal{N}(\mu_{2},\sigma_{2})$,
$X_{1}+X_{2} \sim
\mathcal{N}(\mu_{1}+\mu_{2},\sqrt{\sigma_{1}^{2}+\sigma_{2}^{2}})$
\item $X\sim\mathcal{N}(\mu,\sigma)$,then $X^{2}/\sigma^{2}\sim
  \chi_{1}(\mu^{2}/\sigma^{2})$,$\chi_{1}$为自由度为1的卡方分布
\item $Z=X_{1}X_{2}$,其中$X_{2},X_{2}$为标准正态分布,则Z为“乘积”正态
  分布,pdf为:
\begin{equation}
\label{eq:24}
f_{z}(z)=\pi^{-1}K_{0}(|z|)
\end{equation}
其中$K_{0}$为修正第二类Bessel函数
\begin{shaded}
微分方程:
\begin{equation}
\label{eq:39}
x^{2}\frac{d^{2}y}{dx^{2}}+x\frac{dy}{dx}+(x^{2}-\alpha^{2})y=0
\end{equation}
对于任意的复数$\alpha$,方程的标准解被称为Bessel函数。对于正的$\alpha$
或者整数,如果方程的解在$x=0$处有届,则解被称为第一类Bessel函数,记为
$J_{\alpha}(x)$;如果在$x=0$处为奇异并且多值,则该解被称为第二类Bessel
函数,记为
$Y_{\alpha}(x)$. $H_{\alpha}^{1}(x)=J_{\alpha}(x)+iY_{\alpha}(x),H_{\alpha}^{2}(x)=J_{\alpha}(x)-iY_{\alpha}(x)$
为另外两个线性无关的解,被称为Hankel函数。\\
微分方程:
\begin{equation}
\label{eq:40}
x^{2}\frac{d^{2}y}{dx^{2}}+x\frac{dy}{dx}-(x^{2}+\alpha^{2})y=0
\end{equation}
对应的解,被称为修正的Bessel函数,第一类和第二类分别记录为$I_{\alpha},K_{\alpha}$
\end{shaded}
\item $Z=\sqrt{X_{1}^{2}+X_{2}^{2}}$,其中$X_{1},X_{2}$为标准正态分布,
  则Z为Rayleigh分布
\item $Z=X_{1}\div X_{2}$,其中$X_{1},X_{2}$为标准正态分布,则Z为Cauchy(0,1)分
  布
\item $Z=X_{1}^{2}+X_{2}^{2}+\cdots+X_{n}^{2}$,其中
  $X_{i}\sim\mathcal{N}(0,1)$,则Z为$\chi_{n}$ 自由度为$n$的卡方分布
\item $X_{1},X_{2}\cdots X_{n}\sim
  \mathcal{N}(0,1)$,$Y_{1},Y_{2},\cdots Y_{m}\sim \mathcal{N}(0,1)$,则
  它们的归一化平方和之比:
\begin{equation}
\label{eq:25}
F=\frac{(X_{1}^{2}+X_{2}^{2}+\cdots+X_{n}^{2})/n}{(Y_{1}^{2}+Y_{2}^{2}+\cdots+Y_{m}^{2})/m}\sim F_{n,m}
\end{equation}
其中$F_{n,m}$为自由度为$(n,m)$的F分布。
\end{itemize}
\subsection{Gamma Distribution}
\begin{shaded}
\noindent Notation: Gamma分布具有两种表示,形状-尺度(shape-scale)
\begin{equation}
\label{eq:29}
X\sim\Gamma(k,\theta)=Gamma(k,\theta)
\end{equation}
或者形状-率(shape-rate)
\begin{equation}
\label{eq:30}
X\sim\Gamma(\alpha,\beta)=Gamma(\alpha,\beta)
\end{equation}
其中$\alpha=k,\beta=\frac{1}{\theta}$
\end{shaded}
\noindent PDF:
\begin{equation}
\label{eq:26}
f(x;k,\theta)=\frac{x^{k-1}e^{-\frac{x}{\theta}}}{\theta^{k}\Gamma(k)}
\end{equation}
for $x>0,k,\theta>0$
\begin{equation}
\label{eq:31}
g(x;\alpha,\beta)=\frac{\beta^{\alpha}x^{\alpha-1}e^{-x\beta}}{\Gamma(\alpha)}
\end{equation}
for $x>0,\alpha,\beta>0$. \\
CDF:
\begin{equation}
\label{eq:27}
F(x;k,\theta)=\int_{0}^{x}f(u;k,\theta)du=\frac{\gamma(k,\frac{x}{\theta})}{\Gamma(k)}
\end{equation}
其中$\gamma(.,.)$为不完全gamma函
数:$\gamma(s,x)=\int_{0}^{x}t^{s-1}e^{-t}dt$.
若k为正整数,则gamma分布为Erlang分布.
Summation:
\begin{equation}
\label{eq:28}
\sum_{i=1}^{N}X_{i}\sim Gamma(\sum_{i-1}^{N}k_{i},\theta)
\end{equation}
其中,所有的$X_{i}$相互独立,且$X_{i}\sim Gamma(k_{i},\theta)$,具有相
同的尺度参数。 \\
Scaling:\\
如果$X\sim Gamma(k,\theta)$,对于$c>0$,则有$cX\sim Gamma(k,c\theta)=Gamma(\alpha,\beta/c)$.
\begin{shaded}
事实上,如果$X$为一个指数随机变量,“率”rate参数为$\lambda$,则$cX$为一
个率参数rate参数为$\lambda/c$的随机变量。由于
\begin{equation}
\label{eq:32}
X\sim\mathrm{Exp}(\lambda) =  X\sim Gamma(1,\lambda^{-1})
\quad (k,\theta)\mathrm{-form}
\end{equation}
故有此结论。
\end{shaded}
\subsection{Chi-Square Distribution}
\noindent 如果$Z_{1},Z_{2},Z_{3}\cdots Z_{k}$是独立标准正态随机变量,
则他们的平房和$Q=\sum_{i=1}^{k}Z_{i}^{2}$服从卡方分布,表示为:
\begin{equation}
\label{eq:33}
Q\sim \chi^{2}(k)\quad\text{or}\quad Q\sim\chi_{k}^{2}
\end{equation}
PDF:
\begin{equation}
\label{eq:34}
f(x;k)=\frac{x^{k/2-1}e^{-x/2}}{2^{k/2}\Gamma(\frac{k}{2})} \quad x>0;
\end{equation}
$\Gamma(\frac{1}{2})=\sqrt{\pi}$
\begin{shaded}
当$k=1$时,卡方分布为正态分布的平方;$k=2$时,为参数为$1/2$的指数分
布,Rayleigh分布的平方;$k=3$时,为参数为1的Maxwell分布的平方
$X\sim\text{Maxwell(1) then }X^{2}\sim\chi^{2}(2)$。
\end{shaded}
可加性:独立卡方分布随机变量的和仍旧为卡方分布,其自由度为所有随机变量
自由度之和。\\

如果$X_{1},X_{2}\cdots,X_{n}$是独立的卡方分布随机变量,自由
度分别为$k_{i}$,则$Y=X_{1}+X_{2}+\cdots+X_{n}$为自由度为
$\sum_{i=0}^{n}k_{i}$的卡方分布。
\subsection{Nakagami-m Distribution}
\noindent PDF:
\begin{equation}
\label{eq:37}
f(x;m,\Omega)=\frac{2m^{m}}{\Gamma(m)\Omega^{m}}x^{2m-1}\mathrm{exp}(-\frac{m}{\Omega}x^{2})
\end{equation}
其中,$m,\Omega$可以表示为:
\begin{equation}
\label{eq:38}
m=\frac{\mathbb{E}^{2}[X^{2}]}{\mathrm{Var}[X^{2}]},\quad \Omega=\mathbb{E}[X^{2}]
\end{equation}
Generation:
\begin{itemize}
\item 从Gamma分布产生,给定一个$Y\sim\Gamma(k,\theta)$,通过设置
  $k=m,\theta=\Omega/m$,求取$Y$的根号$X=\sqrt{Y}$,则有
  $X\sim\mathrm{Nakagami}(m,\Omega)$
\item 从Chi-square产生,将卡方分布的自由度$k$变为$2m$,同时通过随机变量
  的尺度变换得到。$Y\sim\chi(2m)$,$X=\sqrt{\Omega/2m}Y$服从参数为
  $(m,\Omega)$的Nakagami分布.
\end{itemize}
\section{Moment-Generating Function}
对于一个随机变量$x$和对应的概率密度函数$f(x)$,如果存在一个h使得在
$|t|<h$时
\begin{equation}
\label{eq:41}
M(t)=\mathbb{E}(e^{tx})
\end{equation}
$M(t)$被称为MGF函数。

如果随机变量$X$和$Y$相互独立,则MGF函数满足:
\begin{equation}
\label{eq:42}
\begin{split}
M_{x+y}(t) & = \mathbb{E}(e^{t(x+y)}) \\
          & = \mathbb{E}(e^{tx}e^{ty}) \\
          & = \mathbb{E}(e^{tx})\mathbb{E}{(e^{ty})} \\
          & = M_{x}(t)M_{y}(t)
\end{split}
\end{equation}
如果$M(t)$在0点可微,则在0点的n阶矩可以用MGF在0点的微分表示:
\begin{equation}
\label{eq:43}
\begin{split}
M(t)=\mathbb{E}(e^{tx}) &\quad M(0)=1 \\
M^{'}(t)=\mathbb{E}(xe^{tx}) &\quad M^{'}(0)=\mathbb{E}(x) \\
M^{''}(t)=\mathbb{E}(x^{2}e^{tx}) &\quad M^{''}(0)=\mathbb{E}(x^{2})
\\
M^{(n)}(t)=\mathbb{E}(x^{n}e^{tx}) &\quad M^{(n)}(0)=\mathbb{E}(x^{n})
\end{split}
\end{equation}

对于实分量的随机变量向量$X$,MGF函数可以表示为:
\begin{equation}
\label{eq:44}
M_{X}(t)=\mathbb{E}(e^{<t,X>})
\end{equation}
其中$t$是一个向量,$<.,.>$表示向量点积.

常见分布的MGF函数:
\begin{table}[htbp]
\centering
\begin{tabular}{|c|c|}
\hline Distribution & Moment-generating function \\
\hline Bernoulli $P(X=1)=p$ & $1-p+pe^{t}$ \\
\hline Geometric $(1-p)^{k-1}p$ & $\frac{pe^{t}}{1-(1-p)e^{t}},\forall
                                  t < -ln(1-p)$ \\
\hline Binomial $B(n,p)$ & $(1-p+pe^{t})^{n}$ \\
\hline Poisson $\mathrm{Pois}(\lambda)$ & $e^{\lambda(e^{t}-1)}$ \\
\hline Uniform continuous $U(a,b)$ & $\frac{e^{tb}-e^{ta}}{t(b-a)}$ \\
\hline Uniform discrete $U(a,b)$ &
                                   $\frac{e^{at}-e^{(b+1)t}}{(b-a+1)(1-e^{t})}$
  \\
\hline Normal $N(\mu,\sigma^{2})$ &
                                    $e^{t\mu+\frac{1}{2}\sigma^{2}t^{2}}$
  \\
\hline Multivariate normal $N(\mu,\Sigma)$ &
                                             $e^{t^{T}\mu+\frac{1}{2}t^{T}\Sigma
                                             t}$ \\
\hline Chi-square $\chi_{k}^{2}$ & $(1-2k)^{-k/2}$ \\
\hline Gamma $\Gamma(k,\theta)$ & $(1-t\theta)^{-k}$ \\
\hline Exponential $\mathrm{Exp}(\lambda)$ & $(1-t\lambda^{-1})^{-1},\quad
                                    t<\lambda$ \\
\hline Cauchy $(\mu,\theta)$ & does not exist \\
\hline Rayleigh($\sigma$) & $1+\sigma
                            te^{\sigma^{2}t^{2}/2}\sqrt{\frac{\pi}{2}}(erf(\frac{\sigma
                            t}{\sqrt{2}})+1)$ \\
\hline
\end{tabular}
\caption{\label{tab:mgf}}
\end{table}
\bibliographystyle{plain}
\bibliography{ref} 
\end{document}
